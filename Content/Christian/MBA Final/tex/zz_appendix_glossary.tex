%=====================================================
% Glossary
%=====================================================

%-----------------------------------------------------
% Samples
%-----------------------------------------------------

% Usage:
% \gls{glos:AD} is pretty interesting. If we have a cross reference from
% the acronyms, we can also directly go to that using \gls{AD}; this
% requires then that over there, we have something like
%  description=\glslink{glos:AD}{Active Directory}

% \newglossaryentry{glos:AD}{
% name=Active Directory,
% description={Active Directory is the directory service for
% Windows based networks, that allows central organization and
% administration of any network resource.
% It allows a single-sign-on concept independent from network
% topologies or network protocols. As a prerequisite you need
% a Windows Server acting as Domain Controller. This computer
% stores all necessary data, e.\,g.~usernames and corresponding
% passwords.}
% }


%-----------------------------------------------------
% Content
%-----------------------------------------------------


\setlength{\abovedisplayskip}{0pt}

%<content>%
% \renewcommand{\theHequation}{\theHsection.\equationgrouping\arabic{equation}}

%=================================================================================
% Break Even Point
%=================================================================================

%\def\glosdescbep{
%The Break Even Point, or BEP, is the point where through a cost volume analysis, the total revenue
%equals the total costs: %\citep[304]{Atrill-2008qf}:
%\begin{subequations}
%\begin{align}
%\text{Total Sales Revenue} &\overset{!}{=} \text{Total Costs} \label{eqn:bep}
%\intertext{and, with}
%\text{Total Sales Revenue} &= \text{Fixed Costs} + \text{Total Variable Costs}
%\intertext{we get, for the number $n$ of produced units at BEP:}
%n \times \text{Sales Revenue per Unit} &= \text{Fixed Costs} + n \times  \text{Variable Costs per Unit}
%\intertext{and solving for $n$, we get the number of units to sell at BEP:}
%\end{align}
%\end{subequations}
%\begin{equation}\label{eqn:bep_contribution}
%\tikzmarkin{bep2}(1.0,-0.75)(-1.0,0.85)
%        \tikz[baseline]{
%            \node[fill=red!20,anchor=base] (t1)
%            {$ n $};
%        } =
%        \frac{\tikz[baseline]{
%           \node[fill=blue!10, anchor=base] (t2)
%            {$\text{Fixed Costs}$};
%        }}{
%        \tikz[baseline]{
%            \node[fill=green!20,anchor=base] (t3)
%            {$\text{Sales Revenue per Unit} - \text{Variable Cost per Unit}$};
%        }}
%\tikzmarkend{bep2}
%\end{equation}
%
%\begin{tikzpicture}[overlay]
%\coordinate (col-aa) at ($(bep2)+(2.5,0.4)$);
%\coordinate (col-ab) at ($(bep2)+(5.5,-2.0)$);
%\node[align=left,right] at (col-aa) {\small{\emph{Number of units produced at BEP}}};
%\node[align=left,left] at (col-ab) {\small{\emph{Contribution}}};
%        \path[red,-stealth,->] (col-aa) edge [bend right] (t1);
%        \path[red,-stealth,->] (col-ab) edge [out=0, in=-90] (t3);
%\end{tikzpicture}
%
%\bigskip \emph{Note:} As the fixed costs are time based, the BEP must be expressed with regards to a period of time.
%}
%\newglossaryentry{glos:bep}{
%  name=Break Even Point (BEP),
%  description={\glosdescbep}
%}

%\def\glosdescprofitability{The attribute of yielding gains or profits. Businesses typically operate with the main objective of generating wealth for their %owners. Profitability ratios offer a glimpse into the extent of success in accomplishing this goal. They illustrate the profit earned (or relevant figures %impacting profit, such as sales revenue or specific expenses like labor costs) in comparison to other vital numbers in the financial statements or a %particular business resource.'' \citep[141]{Atrill:2010ys}
%It is measured as \glslink[hyper=true]{glos:rosf}{\gls{glos:rosf}},  \glslink[hyper=true]{glos:roce}{\gls{glos:roce}},
% \glslink[hyper=true]{glos:netprofitmargin}{\gls{glos:netprofitmargin}} and
%  \glslink[hyper=true]{glos:grossprofitmargin}{\gls{glos:grossprofitmargin}}. 
%}
%\newglossaryentry{glos:profitability}{
%  name=profitability,
%  firstplural=profitabilities,
%  plural=profitabilities,
%  description={\glosdescprofitability}
%}
%
%
%%=================================================================================
%% Gross Profit Margin
%%=================================================================================
%\def\glosdescgrossprofitmargin{``The gross profit margin ratio relates
%the gross profit of the business to the sales revenue generated over the same period
%\dots\, The ratio is therefore a measure of profitability in buying (or producing)
%and selling goods before any other expenses:'' \citep[178]{Atrill:2006ly}
%
%\begin{subequations}
%\begin{align}
%  \text{Gross Profit Margin} &= \frac{\text{Gross Profit}}{\text{Sales Revenue}} 
%  \intertext{and with}
%  \text{Gross Profit} &= \text{Sales Revenue} - \text{\glslink{cogs}{COGS}}
%  \intertext{it follows}
%\tikzmarkin{a}(5.4,-0.5)(-1.0,0.75)\text{Gross Profit Margin} &=  1 - \frac{\text{\gls{cogs}}}{\text{Sales Revenue}} \label{eqn:grossprofitmargin}\tikzmarkend{%a}
%\end{align}
%\end{subequations}
%
%
%Equation \eqref{eqn:grossprofitmargin} features the relation of cost of goods sold to revenue,
%also known as \glslink{glos:costofgoodstosales}{cost of goods sold ratio}. The gross profit
%margin is normally expressed as a percentage.
%}
%\newglossaryentry{glos:grossprofitmargin}{
%  name=Gross profit margin,
%  firstplural=Gross profit margins,
%  plural=Gross profit margins,
%  description={\glosdescgrossprofitmargin}
%}
%
%\def\glosdescliquidity{Liquidity is the ability to pay short-term obligations
%when they fall due. ``The availability of adequate liquid assets is crucial for a business's survival, as these resources are necessary to fulfill upcoming %obligations (i.e., debts that need to be settled in the near future).'
%\citep[141]{Atrill:2010ys} It is measured as
%\glslink[hyper=true]{glos:currentratio}{\gls{glos:currentratio}} or
%\gls[hyper=true]{glos:acidtestratio}.
%}
%\newglossaryentry{glos:liquidity}{
%  name=liquidity,
%  firstplural=liquidities,
%  plural=liquidities,
%  description={\glosdescliquidity}
%}

%=================================================================================
% Operating Profit Margin
%=================================================================================


\def\glosdescoperatingprofitmargin{
The operating profit margin is defined as
\begin{subequations}
\begin{align}
\text{Operating Profit Margin} &= \frac{\text{Operating Profit}}{\text{Sales Revenue}}
\intertext{where}
\text{Operating Profit} &= \text{Gross Profit} - \text{SG\&A} - \text{R\&D}
\intertext{hence}
\begin{split}\text{Operating Profit Margin} &= \text{Gross Profit Margin}\\ & - \frac{\text{SG\&A}}{\text{Sales Revenue}} - \frac{\text{R\&D}}{\text{Sales Revenue}} \label{eqn:opm}\end{split}
\end{align}
\end{subequations}

Looking at equation \eqref{eqn:opm}, the operating margin, also called EBITDA margin, can be identified
as the gross profit margin less the \gls{sga} to sales ratio---the cost of doing business per revenue---and the
R\&D to sales ratio---also referred to as ``research intensity" (see \citeauthor{Hundley:1996kx} with regards
to the relationship between research intensity, profitability and liquidity).
}
\newglossaryentry{glos:operatingprofitmargin}{
  name=Operating Profit Margin,
  firstplural=Operating Profit Margins,
  plural=Operating Profit Margins,
  description={\glosdescoperatingprofitmargin}
}

%=================================================================================
% Gearing
%=================================================================================


%\def\glosdescgearingratio{
%``The gearing ratio measures the contribution of long-term lenders to the
%long-term capital structure of the business:" \citep[192]{Atrill:2006ly}
%
%\begin{equation}
%\text{Gearing ratio} = \frac{\text{Long-term Liabilities}}{\text{Share Capital} + \text{Reserves} + \text{Long-term Liabilities}}
%\end{equation}
%}
%\newglossaryentry{glos:gearingratio}{
%  name=Gearing Ratio,
%  description={\glosdescgearingratio}
%}




%</content>%























